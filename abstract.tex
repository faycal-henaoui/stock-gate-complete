\chapter*{Résumé}
\addcontentsline{toc}{chapter}{Résumé}

La gestion des factures fournisseurs représente un défi majeur pour les PME et cabinets comptables, générant des coûts administratifs élevés et des erreurs de saisie fréquentes. Ce projet de fin d'études propose une solution complète d'automatisation basée sur l'Intelligence Artificielle et distribuée en mode SaaS.

Notre approche repose sur une architecture moderne en microservices combinant une interface Web interactive (React/Node.js) et un moteur d'OCR puissant (Python/FastAPI). Contrairement aux solutions classiques basées sur des modèles rigides (Templates), nous avons développé une méthode hybrique innovante : l'utilisation d'un modèle de Deep Learning de dernière génération, \textbf{SVTR} (Scene Text Recognition with a Single Visual Model), fine-tuné spécifiquement pour la reconnaissance de tickets de caisse, couplé à un algorithme d'extraction spatio-sémantique pour reconstruire la structure des tableaux.

Pour garantir l'intégration comptable, nous avons conçu le module "Smart Matching", un système de réconciliation intelligent utilisant la logique floue (Levenshtein/Jaccard) pour lier les désignations fournisseurs aux produits du stock interne.

Les résultats expérimentaux montrent une précision de reconnaissance de caractère supérieure à \textbf{96\%} et une réduction du temps de traitement par un facteur 30 par rapport à la saisie manuelle. L'étude économique confirme la viabilité du modèle, avec un retour sur investissement estimé à moins de 6 mois pour une PME standard.

\textbf{Mots-clés :} OCR, Deep Learning, SVTR, Extraction d'Information, SaaS, Automatisation Comptable, Smart Matching.

\vspace{1.5cm}

\chapter*{Abstract}
\addcontentsline{toc}{chapter}{Abstract}

Supplier invoice management poses a significant challenge for SMEs and accounting firms, resulting in high administrative costs and frequent data entry errors. This final year project proposes a complete automation solution based on Artificial Intelligence and distributed as a SaaS platform.

Our approach relies on a modern microservices architecture combining an interactive Web interface (React/Node.js) and a powerful OCR engine (Python/FastAPI). Unlike traditional template-based solutions, we developed an innovative hybrid method: leveraging a state-of-the-art Deep Learning model, \textbf{SVTR} (Scene Text Recognition with a Single Visual Model), fine-tuned specifically for receipt recognition, coupled with a spatio-semantic extraction algorithm to reconstruct table structures.

To ensure accounting integration, we designed the "Smart Matching" module, an intelligent reconciliation system using fuzzy logic (Levenshtein/Jaccard) to link supplier descriptions to internal stock products.

Experimental results demonstrate a character recognition accuracy exceeding \textbf{96\%} and a 30-fold reduction in processing time compared to manual entry. The economic study confirms the model's viability, with an estimated return on investment of less than 6 months for a standard SME.

\textbf{Keywords:} OCR, Deep Learning, SVTR, Information Extraction, SaaS, Accounting Automation, Smart Matching.

\chapter*{Introduction Générale}
\addcontentsline{toc}{chapter}{Introduction Générale}

Dans l'ère numérique actuelle, la transformation digitale est devenue un impératif stratégique pour les entreprises cherchant à optimiser leurs processus et à réduire leurs coûts opérationnels. Parmi ces processus, la gestion des achats et de la comptabilité reste souvent entravée par des tâches manuelles répétitives, notamment la saisie des factures fournisseurs.

Le traitement manuel des factures est non seulement chronophage, mais il est également source d'erreurs humaines (inversion de chiffres, erreurs de saisie, pertes de documents). Pour une entreprise gérant des centaines, voire des milliers de factures mensuelles, l'impact sur la productivité et la gestion des stocks est considérable.

Ce projet s'inscrit dans cette optique de modernisation. Notre objectif est de concevoir et réaliser une solution complète d'automatisation capable de lire, comprendre et extraire les informations clés des factures (Fournisseur, Date, Tableaux d'articles) à partir d'images ou de PDF, puis de les intégrer automatiquement dans un système de gestion de stock.

Pour ce faire, nous avons développé une chaîne de traitement hybride alliant la puissance du \textbf{Deep Learning} pour la reconnaissance optique de caractères (OCR) via PaddleOCR \cite{paddleocr}, à des algorithmes d'analyse spatiale innovants pour la structuration des données. Cette solution est intégrée dans une application web moderne (React/Node.js) offrant une interface intuitive pour la validation et le rapprochement automatique des produits (Smart Matching).
